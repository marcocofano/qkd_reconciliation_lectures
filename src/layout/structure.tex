%%%%%%%%%%%%%%%%%%%%%%%%%%%%%%%%
%  Main Structural configuration

% DISCLAIMERS and Acknowledgements
%%%%%%%%%%%%%%%%%%%%%%%%%%%%%%%%%%%%%%%%%%%%%%%%%%%%%%%%%%%%%%%
% The layout and coloring is a modification of Templates used @ LuxQuanta Technologies
% by the Author and his group
% Authors:
% Marco Cofano, Azarias Boutin
%
%%%%%%%%%%%%%%%%%%%%%%%%%%%%%%%%%%%%%%%%%%%%%%%%%%%%%%%%%%%%%%%

%%%%%%%%%%%%%%%%%%%%%%%%%%%%%%%%%%%%%%%%%%%%%%%%%%%%%%%%%%%%%%%
% Some environments take inspiration from: Lachaise Assignment
% Authors:
% Marion Lachaise & François Févotte
% Vel (vel@LaTeXTemplates.com)
%
% License:
% CC BY-NC-SA 3.0 (http://creativecommons.org/licenses/by-nc-sa/3.0/)
%%%%%%%%%%%%%%%%%%%%%%%%%%%%%%%%%%%%%%%%%%%%%%%%%%%%%%%%%%%%%%%

%----------------------------------------------------------------------------------------
%	PACKAGES AND OTHER DOCUMENT CONFIGURATIONS
%----------------------------------------------------------------------------------------

\usepackage{amsmath,amsfonts,stmaryrd,amssymb} % Math packages
\usepackage{bbold}
\usepackage{mathtools}
\usepackage{enumerate} % Custom item numbers for enumerations
% \usepackage{algcompatible}
% \usepackage{algorithmic} % Algorithms
\usepackage[ruled]{algorithm2e} % Algorithms
\usepackage[framemethod=tikz]{mdframed} % Allows defining custom boxed/framed environments
\usepackage{listings} % File listings, with syntax highlighting
\usepackage{tikz}
\usepackage{amsthm}
\usepackage{subfiles}
\usepackage{afterpage}
\usepackage{graphicx}
\usetikzlibrary{shapes,arrows}
\usepackage{geometry} % Required for adjusting page dimensions and margins
\usepackage[utf8]{inputenc} % Required for inputting international characters
\usepackage{lmodern}
\usepackage[T1]{fontenc} % Output font encoding for international characters
\usepackage{titlesec} % The titles
\usepackage{hyperref} % The titles
\usepackage[english]{babel} % The titles
\usepackage[
  acronym,
  nopostdot,
  style=index,
  nonumberlist,
  toc
]{glossaries}
\usepackage[nottoc]{tocbibind}

%----------------------------------------------------------------------------------------
% Custom colors
%----------------------------------------------------------------------------------------
\definecolor{LQBlue}{HTML}{2F5496}
\definecolor{gray}{rgb}{0.5,0.5,0.5}
\definecolor{mauve}{rgb}{0.58,0,0.82}
\definecolor{gray75}{gray}{0.75}


\lstset{frame=tb,
	language=Python,
	aboveskip=3mm,
	belowskip=3mm,
	showstringspaces=false,
	columns=flexible,
	basicstyle={\small\ttfamily},
	numbers=none,
	numberstyle=\tiny\color{gray},
	keywordstyle=\color{LQBlue},
	commentstyle=\color{gray},
	stringstyle=\color{mauve},
	breaklines=true,
	breakatwhitespace=true,
	tabsize=4
}
%------------------------------------------------------------------------
%	RENEW commands
%------------------------------------------------------------------------
\newcommand{\hsp}{\hspace{10pt}}
\renewcommand{\chaptername}{Lecture}

\renewcommand{\thechapter}{\arabic{chapter}} % Chapter numbering
\renewcommand{\thesection}{\arabic{section}} % section numbering
\renewcommand{\thesubsection}{\arabic{subsection}} % subsection numbering
\renewcommand{\thesubsubsection}{\alph{subsubsection}} % subsubsection numbering
\renewcommand{\thefigure}{\arabic{figure}} % Figure numbering

%------------------------------------------------------------------------
%	Theorem Environments
%------------------------------------------------------------------------
\newtheoremstyle{custom}%
  {15pt} % Space above
  {15pt} % Space below
  {\itshape} % Body font
  {} % Indent amount
  {\bfseries\color{LQBlue}} % Theorem head font
  {.} % Punctuation after theorem head
  {5pt plus 1pt minus 1pt} % Space after theorem head
  {} % Theorem head spec

\theoremstyle{custom}
\newtheorem{theorem}{Theorem}[chapter]
\newtheorem{definition}{Definition}[chapter]
\newtheorem*{remark}{Remark}


\tikzstyle{block} = [draw, fill=white, rectangle,
minimum height=3em, minimum width=6em]
\tikzstyle{socket} = [draw, fill=white, rectangle, minimum height=0.5em, minimum width=0.5em]
\tikzstyle{sum} = [draw, fill=white, circle, node distance=1cm]
\tikzstyle{input} = [coordinate]
\tikzstyle{output} = [coordinate]
\tikzstyle{pinstyle} = [pin edge={to-,thin,black}]

%----------------------------------------------------------------------------------------
%	DOCUMENT MARGINS
%----------------------------------------------------------------------------------------


\geometry{
	paper=a4paper, % Paper size, change to letterpaper for US letter size
	top=2.5cm, % Top margin
	bottom=3cm, % Bottom margin
	left=2.5cm, % Left margin
	right=2.5cm, % Right margin
	headheight=14pt, % Header height
	footskip=1.5cm, % Space from the bottom margin to the baseline of the footer
	headsep=1.2cm, % Space from the top margin to the baseline of the header
	%showframe, % Uncomment to show how the type block is set on the page
}

%----------------------------------------------------------------------------------------
%	FONTS
%----------------------------------------------------------------------------------------


\usepackage{XCharter} % Use the XCharter fonts

%----------------------------------------------------------------------------------------
%	COMMAND LINE ENVIRONMENT
%----------------------------------------------------------------------------------------

% Usage:
% \begin{commandline}
%	\begin{verbatim}
%		$ ls
%
%		Applications	Desktop	...
%	\end{verbatim}
% \end{commandline}

\mdfdefinestyle{commandline}{
	leftmargin=10pt,
	rightmargin=10pt,
	innerleftmargin=15pt,
	middlelinecolor=black!50!white,
	middlelinewidth=2pt,
	frametitlerule=false,
	backgroundcolor=black!5!white,
	frametitle={Command Line},
	frametitlefont={\normalfont\sffamily\color{white}\hspace{-1em}},
	frametitlebackgroundcolor=black!50!white,
	nobreak,
}

% Define a custom environment for command-line snapshots
\newenvironment{commandline}{
	\medskip
	\begin{mdframed}[style=commandline]
		}{
	\end{mdframed}
	\medskip
}

%----------------------------------------------------------------------------------------
%	FILE CONTENTS ENVIRONMENT
%----------------------------------------------------------------------------------------

% Usage:
% \begin{file}[optional filename, defaults to "File"]
%	File contents, for example, with a listings environment
% \end{file}

\mdfdefinestyle{file}{
	innertopmargin=1.6\baselineskip,
	innerbottommargin=0.8\baselineskip,
	topline=false, bottomline=false,
	leftline=false, rightline=false,
	leftmargin=2cm,
	rightmargin=2cm,
	singleextra={%
			\draw[fill=black!10!white](P)++(0,-1.2em)rectangle(P-|O);
			\node[anchor=north west]
			at(P-|O){\ttfamily\mdfilename};
			%
			\def\l{3em}
			\draw(O-|P)++(-\l,0)--++(\l,\l)--(P)--(P-|O)--(O)--cycle;
			\draw(O-|P)++(-\l,0)--++(0,\l)--++(\l,0);
		},
	nobreak,
}

% Define a custom environment for file contents
\newenvironment{file}[1][File]{ % Set the default filename to "File"
	\medskip
	\newcommand{\mdfilename}{#1}
	\begin{mdframed}[style=file]
		}{
	\end{mdframed}
	\medskip
}

%----------------------------------------------------------------------------------------
%  HEADINGS
%----------------------------------------------------------------------------------------
\newcommand{\cchapter}[1]{
	\chapter*{#1}
	\phantomsection
	\addcontentsline{toc}{chapter}{#1}
}

% Custom format for numbered chapters
\titleformat{\chapter}[display]
  {\bfseries\LARGE\color{LQBlue}} % Large and bold, with color
  {\chaptername\ \thechapter} % "Lecture 1" with color
  {20pt} % Space between "Lecture 1" and chapter title
  {\Huge\bfseries\color{LQBlue}} % Chapter title in large font and blue

\titlespacing*{\chapter}{0pt}{50pt}{40pt} % Adjust vertical spacing

% Custom format for non-numbered chapters
\titleformat{name=\chapter,numberless}[display]
  {\bfseries\LARGE\color{LQBlue}} % Large and bold, with color
  {} %% No "Lecture" and number
  {20pt} % Space between "Lecture 1" and chapter title
  {\Huge\bfseries\color{LQBlue}} % Chapter title in large font and blue

\titlespacing*{\chapter}{0pt}{50pt}{40pt} % Adjust vertical spacing


% Custom format for numbered sections
\titleclass{\section}{straight}

\titleformat{\section}[hang]
  {\large\normalfont\sffamily} % Section title formatting: large, sans-serif
  {\color{LQBlue}\bfseries\thechapter.\thesection} % "Chapter.Section" in bold and blue
  {1em} % Spacing between the label and the title
  {\bfseries\color{LQBlue}} % Title in bold and blue

\titlespacing*{\section}{0pt}{*2}{*1} % Adjust spacing before and after sections

%Format of a subsection title
\titleclass{\subsection}{straight}
\titleformat{\subsection}[hang]
{}
{ }
{0pt}
{\small \normalfont \sffamily \hspace{2cm} \bfseries \thechapter.\thesection.\thesubsection \hsp \small \normalfont \sffamily \bfseries }


%Format of a sub-subsection
\titleclass{\subsubsection}{straight}
\titleformat{\subsubsection}[hang]
{}
{ }
{0pt}
{\color{darkgray} \normalsize \normalfont \sffamily \hspace{2.5cm} \thesubsubsection ) \normalsize \normalfont \sffamily }


% Space between tables
\renewcommand{\arraystretch}{1.7}

% Do not reset the figure count each chapter
\counterwithout{figure}{chapter}
\counterwithout{table}{chapter}
\makeatletter
\makeatother
%----------------------------------------------------------------------------------------
%	NUMBERED QUESTIONS ENVIRONMENT
%----------------------------------------------------------------------------------------

% Usage:
% \begin{question}[optional title]
%	Question contents
% \end{question}

\mdfdefinestyle{question}{
	innertopmargin=1.2\baselineskip,
	innerbottommargin=0.8\baselineskip,
	roundcorner=5pt,
	nobreak,
	singleextra={%
			\draw(P-|O)node[xshift=1em,anchor=west,fill=white,draw,rounded corners=5pt]{%
				Question \theQuestion\questionTitle};
		},
}

\newcounter{Question} % Stores the current question number that gets iterated with each new question

% Define a custom environment for numbered questions
\newenvironment{question}[1][\unskip]{
	\bigskip
	\stepcounter{Question}
	\newcommand{\questionTitle}{~#1}
	\begin{mdframed}[style=question]
		}{
	\end{mdframed}
	\medskip
}

%----------------------------------------------------------------------------------------
%	WARNING TEXT ENVIRONMENT
%----------------------------------------------------------------------------------------

% Usage:
% \begin{warn}[optional title, defaults to "Warning:"]
%	Contents
% \end{warn}

\mdfdefinestyle{warning}{
	topline=false, bottomline=false,
	leftline=false, rightline=false,
	nobreak,
	singleextra={%
			\draw(P-|O)++(-0.5em,0)node(tmp1){};
			\draw(P-|O)++(0.5em,0)node(tmp2){};
			\fill[black,rotate around={45:(P-|O)}](tmp1)rectangle(tmp2);
			\node at(P-|O){\color{white}\scriptsize\bf !};
			\draw[very thick](P-|O)++(0,-1em)--(O);%--(O-|P);
		}
}

% Define a custom environment for warning text
\newenvironment{warn}[1][Warning:]{ % Set the default warning to "Warning:"
	\medskip
	\begin{mdframed}[style=warning]
		\noindent{\textbf{#1}}
		}{
	\end{mdframed}
}

%----------------------------------------------------------------------------------------
%	INFORMATION ENVIRONMENT
%----------------------------------------------------------------------------------------

% Usage:
% \begin{info}[optional title, defaults to "Info:"]
% 	contents
% 	\end{info}

\mdfdefinestyle{info}{%
topline=false, bottomline=false,
leftline=false, rightline=false,
nobreak,
singleextra={%
\fill[LQBlue](P-|O)circle[radius=0.6em];
\node at(P-|O){\color{white}\scriptsize\bf i};
\draw[line width=1.5mm, LQBlue](P-|O)++(0,-0.8em)--(O);%--(O-|P);
}}

% Define a custom environment for information
\newenvironment{info}[1][Info:]{ % Set the default title to "Info:"
	\medskip
	\begin{mdframed}[style=info]
		\noindent{\textbf{#1}}
		}{
	\end{mdframed}
}

\mdfdefinestyle{example}{%
topline=false, bottomline=false,
leftline=false, rightline=false,
nobreak,
singleextra={%
\fill[LQBlue](P-|O)circle[radius=0.6em];
\node at(P-|O){\color{white}\scriptsize\bf ex};
\draw[line width=1.5mm, LQBlue](P-|O)++(0,-0.8em)--(O);%--(O-|P);
}}

% Define a custom environment for information
\newenvironment{example}[1][Example:]{ % Set the default title to "Info:"
	\medskip
	\begin{mdframed}[style=example]
		\noindent{\textbf{#1}}
		}{
	\end{mdframed}
}

% -------------------------------------------------
% Custom math operators
% -------------------------------------------------

\DeclareMathOperator{\Tr}{Tr}


% -------------------------------------------------
% Quantum Mechanical stuff
% -------------------------------------------------

\DeclarePairedDelimiter\bra{\langle}{\rvert}
\DeclarePairedDelimiter\ket{\lvert}{\rangle}
\DeclarePairedDelimiterX\braket[2]{\langle}{\rangle}{#1\,\delimsize\vert\,\mathopen{}#2}

\newcommand{\density}[2]{\ket{#1} \bra{#2}}
