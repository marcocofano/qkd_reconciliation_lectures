\chapter{Message Passing Decoding}

\section{Decoding Repetition codes over symmetric channels}
As a warm up for more complex decoding implementations we look at the very simple repetition codes. We implement the decoding algorithm over the BSC and BiAWGN channel. 

We denote the repetition code $R(n, 2)$ over BSC, with channel parameter $\delta$. The minimum distance we have found before is $n$. This code can correct $\frac{n-1}{2}$ errors. The message plus repetition bits are denoted by $(u, x_1, \dots, x_n)$ 

We assume that at the receiving end the $y$ string contains $\gamma_0$ 0s and $\gamma_1$ 1s. Then:
\begin{eqnarray}
	d(0, \dots, 0, y) &=& \gamma_1  = n-\gamma_0 \\
	d(1, \dots, 1, y) &=& \gamma_0
\end{eqnarray}


\section{Factor graphs}

\section{Marginalization via message passing on trees}

\section{Cycles, Trees and optimality of the decoder}
